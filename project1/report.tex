\documentclass[11pt]{article}
\usepackage{amsmath}
\usepackage{amsfonts}
\usepackage{amsthm}
\usepackage[utf8]{inputenc}
\usepackage[margin=0.75in]{geometry}

\title{CSC111 Winter 2025 Project 1}
\author{Ivan Xie and Saithan Panchaharan}
\date{\today}

\begin{document}
\maketitle

\section*{Running the game}
We should be able to run your game by simply running \texttt{adventure.py}. If you have any other requirements (e.g., installing certain modules), describe them here. Otherwise, skip this section.

\section*{Game Map}
Example game map below (edit it to show your actual game map):

\begin{verbatim}
  1  2  3
  4  -1 -1
  5  6  7
\end{verbatim}

Starting location is: 1

\section*{Game solution}
List of commands: win_walkthrough = ["no", "go east", "pick up", "go west", "go south", "no"
                       "pick up", "go south", "no", "go east", "go east", "submit"]
                  win_walkthrough2 = ["no", "pick up", "go east", "pick up", "go east", "no", "drop",
                  "phone", "go west", "go west", "go south", "no", "pick up", "go south",
                  "no", "go east", "pick up", "go west", "drop", "wallet", "go east", "go east", "submit"]

There are 2 win conditions in this game
1st win condition (win_walkthrough):
    - You make it back to the dorm room with all the required items within the required time and moves
2nd win condition (win_walkthrough2):
    - You make it back to the dorm room with all the required items within the required moves, but you exceed the time limit, but you also accumulated enough points to still win.

\section*{Lose condition 1}
Description of how to lose the game:
1st way to lose the game:
    - There is a time value in minutes. It is set to 150.
    - Each action the player does has an associated time with it.
    - In this case, the commands in "available_commands" account for 10 minutes. "look" and "inventory account for 5 minutes. The rest are 0.

List of commands: lose_demo = ["no", "pick up", "go east", "pick up", "go east", "no"
                 "drop", "quit", "go west", "go west", "go south", "no"
                 "pick up", "go south", "no", "go east", "pick up",
                 "go west", "drop", "quit", "go east", "go east", "submit"]

Which parts of your code are involved in this functionality:
    - At the end of the loop, the time is updated accordingly using the update_time() function
    - The lose case is handled in the submit() function

\section*{Lose condition 2}
Description of how to lose the game:
2nd way to lose the game:
    - There is a maximum amount of moves the player gets
    - If they exceed it, they lose.

List of commands: lose_demo = ["no", "pick up", "pick up", "pick up", "pick up", "pick up", "pick up",
                               "pick up", "pick up", "pick up", "pick up", "pick up", "pick up", "pick up",
                               "pick up", "pick up", "pick up", "pick up", "pick up", "pick up", "pick up",
                               "pick up", "pick up", "pick up", "pick up", "pick up", "pick up"]

Which parts of your code are involved in this functionality:
    - In the AdventureGame class, the data attribute has a value with the total moves. It is set to 25
    - In the while loop in adventure.py, it checks if you have more than 0 moves and updates the amount of moves you have after each loop

\section*{Lose condition 3}
Description of how to lose the game:
3nd way to lose the game:
    - You can drop items
    - If you drop an item that is required for submission, you cannot get it back and you can never win.

List of commands: lose_demo = ["no", "go east", "pick up", "go east", "no", "drop", "USB Drive"]
Note: You don't lose straight away, but because you dropped a required item, you are never able to win and eventually you will run out of moves and lose.
Note: You can also "undo" but if you do not do it right away it won't matter. Since undo only undoes the most recent event.

Which parts of your code are involved in this functionality:
    - The drop_item() function handles the dropping feature
    - The undo() function handles the undo feature

\section*{Inventory}

\begin{enumerate}
\item All location IDs that involve items in the game:

\item Item data:
\begin{enumerate}
    \item For Item 1:
    \begin{itemize}
    \item Item name: Phone
    \item Item start location ID: 1
    \item Item target location ID: 3
    \end{itemize}
        \item For Item 2:
    \begin{itemize}
    \item Item name: USB Drive
    \item Item start location ID: 2
    \item Item target location ID: 7
    \end{itemize}
        \item For Item 3:
    \begin{itemize}
    \item Item name: Laptop Charger
    \item Item start location ID: 2
    \item Item target location ID: 7
    \end{itemize}
        \item For Item 4:
    \begin{itemize}
    \item Item name: Lucky Mug
    \item Item start location ID: 4
    \item Item target location ID: 7
    \end{itemize}
        \item For Item 5:
    \begin{itemize}
    \item Item name: Wallet
    \item Item start location ID: 6
    \item Item target location ID: 5
    \end{itemize}
    % Copy-paste the above if you have more items, to list ALL items
\end{enumerate}

    \item Exact command(s) that should be used to pick up an item (choose any one item for this example), and the command(s) used to use/drop the item (can copy the list you assigned to \texttt{inventory\_demo} in the \texttt{project1\_simulation.py} file)
        Phone_demo = ["no", "inventory", "pick up", "inventory", "quit"]
    \item Which parts of your code (file, class, function/method) are involved in handling the \texttt{inventory} command:
        - The AdventureGame class has a data attribute that has a list of items as one of its values. This value is current inventory of the player.
        - Functinos that involve the inventory are: submit(), drop_item(), remove_item(), pickup_items(), display_inventory().
\end{enumerate}

\section*{Score}
\begin{enumerate}

    \item Briefly describe the way players can earn scores in your game. Include the first location in which they can increase their score, and the exact list of command(s) leading up to the score increase:

        - If players drop off one of the additional items (wallet, phone) at the target location, they will earn a certain
        amount of points, based on the item
        - At certain locations, players will be given the choice to unscramble words, and doing so correctly will earn them
        5 points

    \item Copy the list you assigned to \texttt{scores\_demo} in the \texttt{project1\_simulation.py} file into this section of the report:
        scores_demo = ["no", "score", "pick up", "go east", "go east", "drop", "phone", "score", "quit"]

    \item Which parts of your code (file, class, function/method) are involved in handling the \texttt{score} functionality:
        - The AdventureGame class has a data attribute that has the score as one of its values
        - The function display_score displays the score
        - There is no single function that updates the score, it is just manually added wherever there is an action that rewards points
        - These functions are: remove_item() and encountered_puzzle()
\end{enumerate}

\section*{Enhancements}
\begin{enumerate}
    \item Describe your enhancement \#1 here
    \begin{itemize}
        \item Brief description of what the enhancement is (if it's a puzzle, also describe what steps the player must take to solve it):
            - The player is given a scrambled word and they have to unscramble it in order to solve the puzzle
            - They get 3 attempts
            - If they manage to guess it within 3 tries, they are rewarded points
        \item Complexity level (choose from low/medium/high):
            - Medium
        \item Reasons you believe this is the complexity level (e.g., mention implementation details, how much code did you have to add/change from the baseline, what challenges did you face, etc.)
            - I believe the complexity level of this enhancement is medium because it involves making a request to an API
            - In terms of code, there was very little to write, but it required some learning of some topics outside of the course
    \end{itemize}

    % Uncomment below section if you have more enhancements; copy-paste as needed
    %\item Describe your enhancement here
    %\begin{itemize}
    %    \item Basic description of what the enhancement is:
    %    \item Complexity level (low/medium/high):
    %    \item Reasons you believe this is the complexity level (e.g., mention implementation details)
    %\end{itemize}
\end{enumerate}


\end{document}
